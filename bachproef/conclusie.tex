%%=============================================================================
%% Conclusie
%%=============================================================================

\chapter{Conclusie}
\label{ch:conclusie}

% TODO: Trek een duidelijke conclusie, in de vorm van een antwoord op de
% onderzoeksvra(a)g(en). Wat was jouw bijdrage aan het onderzoeksdomein en
% hoe biedt dit meerwaarde aan het vakgebied/doelgroep? 
% Reflecteer kritisch over het resultaat. In Engelse teksten wordt deze sectie
% ``Discussion'' genoemd. Had je deze uitkomst verwacht? Zijn er zaken die nog
% niet duidelijk zijn?
% Heeft het onderzoek geleid tot nieuwe vragen die uitnodigen tot verder 
%onderzoek?

%\lipsum[76-80]

In tegenstelling tot wat er vaak gesuggereerd wordt, is de IBM mainframe nog sterk aan innovatie onderheven. Dat kan geconcludeerd worden uit het artikel van \textcite{Almekinders2022}. De introductie van de nieuwe z16 wijst er wederom op dat een mainframe nog heel wat jaren mee kan met moderne technologieën. Door middel van een Telum-processor die aan AI-inferencing doet, kan de nieuwe mainframe fraude detecteren voor het zich heeft voorgedaan. AI-modellen worden getraind door data in Db2-tabellen. Een voorbeeld van een use case waarbij dat kan worden ingezet is volgens \textcite{Sarkar2020} het versnellen van de leningafsluitingen en het bepalen van risico's bij transacties. Deze mainframe is ook klaar voor de nadelen die quantum computing met zich zal meebrengen. Volgens \textcite{Almekinders2022} heeft de z16 mainframe geïmplementeerde wiskundige probleemstellingen die ervoor zorgen dat cryptografie veilig gesteld wordt. Cryptografische gevallen kunnen namelijk makkelijk doorbroken worden door quantum gebaseerde gevallen. De conclusie die in dit onderzoek kan worden getrokken uit de onderzoeken van \citeauthor{Almekinders2022}, \citeauthor{Sarkar2020} en \citeauthor{Cammarota2020} is dat de nieuwste mainframe klaar is om het op te nemen tegen de hedendaagse problemen zoals fraude en de gevaren van quantum computing. Laat dat nu net iets zijn waar gedistrueerde systemen nog niet tegen bestemd zijn. 

Hoe is de hedendaagse mainframe nu vergelijkbaar met gedistribueerde systemen zoals Google Cloud Platform? Volgens \textcite{Sarkar2020} is het al sinds de jaren 60 mogelijk om meerdere systemen te laten draaien die tegelijk dezelfde data kunnen raadplegen. Door middel van de parallel sysplex technologie is dat mogelijk. Mainframes bieden ook manieren om te ontwikkelen met moderne technologieën zoals Python. In kader van dit onderzoek is er een Guide Shared Europe (GSE) bijgewoond in Almere, Nederland waar er een volledige Customer Informational Control System (CICS) transactie werd aangeroepen via een REST API in Python. Dezelfde data was zichtbaar in de browser in plaats van in een CICS-terminal. Tot slot werd er gebruik gemaakt van de Visual Studio Code IDE als tool om datasets te raadplegen en code te schrijven. Hieruit is gebleken dat de mainframe niet zo verouderd is als veel migratiespecialisten zoals Astadia beweren. Volgens \autocite{Astadia2021} is Google Cloud Platform een aantrekkelijker en populairder platform. Zij vormen COBOL of PL/1  om naar Java of C\#, maar tijdens de GSE werd een REST-API gebouwd met Python.Terwijl er geen code-refactoring van het originele programma nodig was. Python is daarnaast een jonge programmeertaal die steeds meer aan populariteit wint.  Het is mogelijk om aan moderne server-client computing te doen maar met de RAS-factoren die mainframe biedt en een gezonde mix van hedendaagse populaire programmeertalen zoals Python. Hieruit wordt in het onderzoek besloten Terwijl er geen code-refactoring van het originele programma nodig was dat mainframes en gedistribueerde systemen nog steeds op hetzelfde niveau staan op vlak van hedendaagse technologieën. 

In het kader van het onderzoek werd een enquête opgesteld om na te gaan of bedrijven met grote workloads plannen hebben om uw mainframe te moderniseren of te migreren. De resultaten van deze enquête waarin enigzins negatiever dan verwacht wat de toekomst van de mainframe betreft. Drie op vijf ondervraagde organisaties gaf aan plannen te hebben om hun workloads te migreren. Deze resultaten zijn af te lezen in figuur \ref{fig: migratieplannen}. Daarnaast is er bij één bedrijf de mogelijkheid dat zij een migratie zullen doen in de toekomst. In combinatie met de resultaten uit figuur \ref{fig: expertisetekort} die aankaarten hoe problematisch het expertisetekort is kunnen we in dit onderzoek vaststellen dat bedrijven hierdoor genoodzaakt zijn om een migratie uit te voeren. Daartegenover stellen we vast dat bedrijven hard hun best doen om deze expertisetekorten op te vangen. 

Een migratie van grote workloads gebeurt niet van de ene dag op de andere. Dat zorgt ervoor dat de vereiste mainframekennis nog steeds een must is binnen deze organisaties. Hiervoor bieden zij interne opleidingstrajecten. Uit het onderzoek van \textcite{2020} kunnen we concluderen dat ook het academisch opleidngsaanbod problematisch is. Met oog op de toekomst van de mainframe zijn er te weinig initiatieven door hogescholen en universiteiten om studenten de juiste mainframeskills aan te leren. De hogeschool Gent is de enige hogeschool in de benelux dat een keuzepakket mainframe-expert aanbiedt aan de studenten. Als gevolg is er een schaarste waaruit enkel kan geconcludeerd worden dat bedrijven binnen een aantal jaar geen andere optie zullen hebben dan te migreren naar een alternatief platform. 

