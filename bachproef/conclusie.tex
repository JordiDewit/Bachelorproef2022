%%=============================================================================
%% Conclusie
%%=============================================================================

\chapter{Conclusie}
\label{ch:conclusie}

% TODO: Trek een duidelijke conclusie, in de vorm van een antwoord op de
% onderzoeksvra(a)g(en). Wat was jouw bijdrage aan het onderzoeksdomein en
% hoe biedt dit meerwaarde aan het vakgebied/doelgroep? 
% Reflecteer kritisch over het resultaat. In Engelse teksten wordt deze sectie
% ``Discussion'' genoemd. Had je deze uitkomst verwacht? Zijn er zaken die nog
% niet duidelijk zijn?
% Heeft het onderzoek geleid tot nieuwe vragen die uitnodigen tot verder 
%onderzoek?

%\lipsum[76-80]

In tegenstelling tot wat er vaak gesuggereerd wordt, is de IBM mainframe nog sterk aan innovatie onderheven. Dat kan geconcludeerd worden uit het artikel van \textcite{Almekinders2022}. De introductie van de nieuwe z16 wijst er wederom op dat een mainframe nog heel wat jaren mee kan met moderne technologieën. Door middel van een Telum-processor die aan AI-inferencing doet, kan de nieuwe mainframe fraude detecteren voor het zich heeft voorgedaan. AI-modellen worden getraind door data in Db2-tabellen. Een voorbeeld van een use case waarbij dat kan worden ingezet is volgens \textcite{Sarkar2020} het versnellen van de leningafsluitingen en het bepalen van risico's bij transacties. Deze mainframe is ook klaar voor de nadelen die quantum computing met zich zal meebrengen. Volgens \textcite{Almekinders2022} heeft de z16 mainframe geïmplementeerde wiskundige probleemstellingen die ervoor zorgen dat cryptografie veilig gesteld wordt. Cryptografische gevallen kunnen namelijk makkelijk doorbroken worden door quantum gebaseerde gevallen. De conclusie die in dit onderzoek kan worden getrokken uit de onderzoeken van \textcite{Almekinders2022}, \textcite{Sarkar2020} en \textcite{Cammarota2020} is dat de nieuwste mainframe klaar is om het op te nemen tegen de hedendaagse problemen zoals fraude en de gevaren van quantum computing. Laat dat nu net iets zijn waar gedistrueerde systemen nog niet tegen bestemd zijn. 

Hoe is de hedendaagse mainframe nu vergelijkbaar met gedistribueerde systemen zoals Google Cloud Platform? Volgens \textcite{Sarkar2020} is het al sinds de jaren 60 mogelijk om meerdere systemen te laten draaien die tegelijk dezelfde data kunnen raadplegen. Door middel van de parallel sysplex technologie is dat mogelijk. Mainframes bieden ook manieren om te ontwikkelen met moderne technologieën zoals Python. In kader van dit onderzoek is er een Guide Shared Europe (GSE) bijgewoond in Almere, Nederland waar een volledige Customer Informational Control System (CICS) transactie werd aangeroepen via een REST API in Python. Dezelfde data was zichtbaar in de browser in plaats van in een CICS-terminal. Tot slot werd er gebruik gemaakt van de Visual Studio Code IDE als tool om datasets te raadplegen en code te schrijven. Hieruit is gebleken dat de mainframe niet zo verouderd is als veel migratiespecialisten zoals Astadia beweren. Volgens \textcite{Astadia2021} is Google Cloud Platform een aantrekkelijker en populairder platform. Zij vormen COBOL of PL/1  om naar Java of C\#, maar tijdens de GSE werd een REST-API gebouwd met Python.Terwijl er geen code-refactoring van het originele programma nodig was. Python is daarnaast een jonge programmeertaal die steeds meer aan populariteit wint.  Het is mogelijk om aan moderne server-client computing te doen maar met de RAS-factoren die mainframe biedt en een gezonde mix van hedendaagse populaire programmeertalen zoals Python. In dit onderzoek werd besloten dat mainframes en gedistribueerde systemen nog steeds op hetzelfde niveau staan op vlak van hedendaagse technologieën. 

In het kader van het onderzoek werd een enquête opgesteld om na te gaan of bedrijven met grote workloads plannen hebben om hun mainframe te moderniseren of te migreren. De resultaten van deze enquête waren enigzins negatiever dan verwacht wat de toekomst van de mainframe betreft. Drie op vijf ondervraagde organisaties gaf aan plannen te hebben om hun workloads te migreren. Deze resultaten zijn af te leiden in figuur \ref{fig: migratieplannen}. Daarnaast is er bij één bedrijf de mogelijkheid op een migratie in de toekomst. In combinatie met de resultaten uit figuur \ref{fig: expertisetekort}, die aankaarten hoe problematisch het expertisetekort is, kan er in dit onderzoek vastgesteld worden dat bedrijven hierdoor genoodzaakt zijn om te migreren. Daartegenover wordt vastgesteld dat bedrijven hun best doen om deze expertisetekorten op te vangen. 

Een migratie van grote workloads gebeurt niet van de ene dag op de andere. Dat zorgt ervoor dat de vereiste mainframekennis nog steeds een noodzaak is binnen deze organisaties. Hiervoor bieden zij interne opleidingstrajecten. Uit het online artikel van \textcite{2020} wordt geconcludeerd dat ook het academisch opleidingsaanbod problematisch is. Met oog op de toekomst van de mainframe zijn er te weinig initiatieven door hogescholen en universiteiten om studenten de juiste mainframeskills aan te leren. Hogeschool Gent is de enige hogeschool in de Benelux die een keuzepakket mainframe-expert aanbiedt aan de studenten. Als gevolg is er een schaarste waaruit enkel kan geconcludeerd worden dat bedrijven binnen een aantal jaren geen andere optie zullen hebben dan te migreren naar een alternatief platform. Opmerkelijk was dat de bedrijven geen specifiek alternatief hebben benoemd maar dat ze enkel het ruime begrip cloud aankaartten. 

In dit onderzoek werd de praktische kant van een migratie onder de loep genoemen. De fictieve workload was wel eens waar een onrealistische verzameling van code en dependencies. Het ging om een 'Hello world' COBOL-applicatie met geen afhankelijkheden zoals copybooks, Db2 tabellen of andere bestanden. Om deze applicatie uitvoerbaar te maken op het AWS- platform waren een handvol complexe stappen vereist waaruit kan geconcludeerd worden dat dit problematisch zou zijn bij real life workloads. Allicht is het een vereiste om de complexiteit te behandelen zoals in het onderzoek van \textcite{Orban2016} wordt aangegeven. Volgens hem is het van essentieel belang dat er bij een migratie een sortering wordt gedaan volgens complexiteit van de workload. Dat kan in dit onderzoek worden bevestigd. Naar aanleiding van de ervaringen die werden opgedaan tijdens de technische uitwerking, kan besloten worden dat een goed plan zoals dat van \textcite{Marble2017} een fundamentele noodzaak is om het doel te bereiken. Als er wordt gekeken naar het rapport van \textcite{Bucchiarone2018} waarin de ervaring van een real life migratie naar Microservices wordt beschreven, benadrukken zij eveneens dat de mainframe loskoppelen een project van lange adem is en nog lang zal blijven. 

\textcite{Astadia2021} beweert dat Google Cloud Platform de toekomst is. Deze instelling, die gespecialiseerd is in het migreren van mainframeworkloads naar het eerder benoemde Google Cloud, ziet dat als de enige behandeling die hedendaagse mainframeworkloads verdienen. Wat tegenstrijdig is met wat ze in het onderzoek van \textcite{Allison2016} beweren. Volgens het onderzoek van \textcite{Allison2016} zijn mainframeworkloads absoluut niet geschikt voor een migratie naar de cloud. Dat onderzoek wijst uit dat de reden hiervoor het besturingssysteemomgeving is. Daarnaast is er naast het besturingssysteem en complexiteit het gebrek aan documentatie dat voor een obstakel zorgt \autocite{Zachry2001}. In dit onderzoek werd geen enkel geslaagde migratie gevonden die Astadia tot nu heeft verwezenlijkt. Als gevolg hiervan wordt de 'sales-talk' van Astadia in twijfel getrokken. Zij kaarten herhalend het prijsverschil aan met de mainframe. Echter worden er nooit over concrete verschilbedragen of prijzen gesproken. Hieruit wordt in dit onderzoek geconcludeerd dat het zogenaamde kostenplaatje enkel en alleen speculaties zijn maar geen harde bewijzen van het verschil. Tijdens de bevraging werd de vraag over het kostenplaatje echter steeds genegeerd. Dat kan dan ook een aanzet zijn voor een vervolg op dit onderzoek. Een andere aanzet kan ook een onderzoek zijn naar alternatieve instellingen dan Astadia die zich bezighouden met het moderniseren (migreren) van mainframeworkloads en wat hun slagingspercentage hierbij is. 

Dit onderzoek kan interessant zijn voor bedrijven die aan het wikken of wegen zijn om hun mainframeworkloads te migreren. Hieruit kunnen zij zelf hun conclusie trekken of het een goed idee is om te migreren. De titel van dit onderzoek bevat het woord moderniseren, waar migratie een type van is. Het is mogelijk om eens over een modernisatie na te denken die niets met een migratie te maken heeft. Er zijn eindeloos veel mogelijkheden om mainframe-applicaties te laten werken met het client-server model in het achterhoofd, het gebruik van REST API's, modernere programmeertalen, enzovoort. Het is mogelijk om aan modernisatie te doen zonder het verlies van de RAS-factoren, code-refactoring, de migratiekosten en het waarschijnlijke jarenplan dat op tafel ligt. 




