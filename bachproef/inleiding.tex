% !TeX spellcheck = de_DE
%%=============================================================================
%% Inleiding
%%=============================================================================

\chapter{\IfLanguageName{dutch}{Inleiding}{Introduction}}
\label{ch:inleiding}
Al 25 jaar lang, beweren onderzoekers dat andere platformen het einde zullen betekenen voor de mainframe. Echter zijn er maar weinig bedrijven met enorme workloads die de stekker uit hun mainframe durven te halen. Dagdagelijks zorgen 220 miljard lijnen COBOL of PL/1-code ervoor dat alle systemen binnen een organisatie blijven functioneren \autocite{Scannell2017}. De mainframe is verantwoordelijk voor miljarden transacties per dag. Denk hierbij aan een bestelling plaatsen op het internet aan de hand van bancontact. Mensen zijn zonder het zelf te beseffen dagelijks eindgebruiker van de mainframe, doch weten veel mensen niet wat een mainframe is. 

\section{\IfLanguageName{dutch}{De grote vraag: wat is een mainframe?}{The big question: what is a mainframe?}}
Een mainframe is simpelweg de grootste vorm van een server. Het is een machine die duizenden applicaties en I/O toestellen kan ondersteunen op een simultane manier. Dit doet een mainframe op basis van de RAS factoren. De reliability, availability en serviceability bij het verwerken van grote workloads zorgt ervoor dat miljarden gebruikers op een performante, veilige en betrouwbare manier kunnen interageren met hun data. Een mainframe moet gezien worden als de centrale datarepository. Het is een gecentraliseerd systeem dat gelinkt wordt aan minder krachtige machines zoals PC's.  Het voordeel van een gecentraliseerd systeem is dat de eindgebruikers hun businessdata maar één keer hoeven te updaten. Bij gedistribueerde systemen moet dit wil en kan dit uiteindelijk zorgen voor corruptie van de data. Dit bewijst de eerste factor, reliability van de mainframe. 

De bekendste producent is International Business Machines Corporation (IBM). Zij hebben een hele grote invloed op de evolutie van wat we vandaag begrijpen onder mainframe. In de jaren vijftig begon de ontwikkeling van Big Iron onder stoom te komen. IBM kwam er met de IBM 705 in 1954. Nadien volgde de IBM 1401 in 1959, maar de doorbraak kwam er pas in 1964. In dat jaar werd de computergeschiedenis helemaal omvergeblazen en was er een revolutie gestart waar we de technologie van vandaag aan te danken hebben. De IBM System/360 zag voor het eerst het daglicht. Het was de eerste mainframe die hardware en software gestandaardiseerd had voor de eindgebruikers. Deze machine kon zowel wetenschappelijk als commercieel worden ingezet. Het was een kwestie van de juiste programma’s te voorzien om specifieke doeleinden de bereiken. De keuze voor de naam heeft een achterliggende betekenis. De 360 in de naam staat voor het feit dat er een 360 graden aan mogelijke use cases bestaat waar deze mainframe kon bij ingezet worden.

De architectuur in de computerwetenschappen beschrijft de organisatorische structuur van een systeem. Deze structuur kan opgedeeld worden in kleine schakelingen die op elkaar zijn afgestemd om perfect te werken. Op vlak van architectuur zijn mainframes door de jaren heen het meest stabiel, compatibel en veilig gebleven doch enorm geëvolueerd. 

Laat ons even teruggaan naar de vraag: Wat is een mainframe? Die term kan beschreven worden als een stijl van operatie, applicatie en besturingssysteem voorzieningen. Een mainframe is wat de business nodig heeft om het hosten te voorzien van: commerciële databanken, transactieservers en applicaties. Dat laatste zijn menigmaal applicaties die een grotere veiligheid en beschikbaarheid eisen, dan wat kleinschalige machines kunnen bieden. 

De eerste mainframesystemen waren enorme, metalen kasten. Zo een metalen kast nam in de beginjaren van de mainframe ongeveer 200 tot 1000 vierkante meter in. Deze vereisten een hoge hoeveelheid aan elektriciteit en koeling. Een typisch datacenter had meerdere mainframe-installaties met een groot aantal I/0-toestellen die geconnecteerd werden aan deze mainframes. Rond 1990 werden mainframeprocessoren en hun I/O-toestellen fysiek kleiner. Echter nam de functionaliteit en capaciteit niet af. In tegendeel, mainframes werden alsmaar krachtiger. De dag van vandaag hebben moderne mainframes de grootte van een koelkast. In sommige gevallen is het nu mogelijk om het mainframes besturingssysteem te draaien op een PC, waardoor het mogelijk is om een mainframe te emuleren. Deze emulators zijn handig voor het ontwikkelen en testen van businessapplicaties voor ze worden gepromoveerd naar het mainframes productiesysteem. 

\section{\IfLanguageName{dutch}{Wie gebruikt mainframecomputers?}{Who uses mainframe computers?}}
Iedereen is eindgebruiker van een mainframecomputer. Op een bepaald moment zijn we allemaal in contact gekomen met een mainframe. Wie als eens geld van een geldautomaat heeft gehaald of een terugbetaling heeft ontvangen van zijn of haar ziekenfonds. Vandaag spelen mainframes een centrale rol in de dagelijkse operaties van werelds grootste bedrijven. De mainframe is namelijk een gigantische pion in de wereld van e-business en e-commerce omgevingen. Eveneens banken, verzekeringsinstellingen, overheidsinstellingen en een groot aantal private bedrijven is de mainframe een ondersteunende kracht voor heel wat businessoperaties. Tot het midden van de jaren negentig waren mainframes het enige aanvaardbare middel om gegevensverwerkingsvereisten van een groot bedrijf aan te pakken. Deze vereisten bestonden dan uit grote en complexe batch jobs zoals salaris- en grootboekverwerking. 

Het vertrouwen in de mainframecomputer is te wijten aan de betrouwbaarheid en stabiliteit dat resulteert in een stabiele technologie. Geen enkele andere computerarchitectuur heeft zo’n evolutionaire verbeteringen doorgemaakt in de wetenschap zijn compatibiliteit met vorige versies te behouden. Door deze eigenschappen worden mainframes het meest gebruikt door instellingen en informatietechnologie organisaties om hun belangrijkste applicaties te hosten. 


\section{\IfLanguageName{dutch}{Probleemstelling}{Problem Statement}}
\label{sec:probleemstelling}

De mainframe heeft er over de jaren heen heel wat concurrenten bijgekregen. Technologieën zoals Amazon Web Services, Google Cloud G4 Platform en Microservices zijn onverbiddelijk de tegenpool van de mainframecomputer. Ook de schaarste in mainframe-experten door het afnemende opleidingsaanbod is een probleem aan het worden. De grote bedrijven die een grote hoeveelheid batchverwerking hebben worden meer en meer geconfronteerd met het in pensioen treden van hun mainframe-experten. Daardoor is er een vergelijkende studie en een onderzoek nodig waaruit kan worden geconcludeerd of de mainframe binnen grote organisaties een rooskleurige toekomst tegemoet gaat of niet. Het kostenplaatje van zo’n migratie is een belangrijke factor. Eveneens de complexiteit die een transitie van workloads met zich meebrengt.

\section{\IfLanguageName{dutch}{Onderzoeksvraag}{Research question}}
\label{sec:onderzoeksvraag}

“Zijn alternatieve systemen (gedistribueerde) beter dan de moderne mainframe?” is een duidelijke vraag die we ons hier kunnen stellen. Anderzijds is “Hebben instanties met grote batchverwerking plannen om de mainframe te vervangen of moderniseren naar gedistribueerde systemen?” ook een vraag die aansluit bij het onderwerp van deze bachelorproef. Uit het antwoord op deze vragen moet een duidelijke conclusie kunnen getrokken worden. Deze conclusie zal ons leren over de mainframe het moderniseren waard is of niet. We kunnen deze vragen natuurlijk nog opsplitsen in verschillende deelvragen. Factoren zoals kostenplaatje, complexiteit en betrouwbaarheid zijn onderwerpen die we in vraag moeten stellen om een duidelijke onderbouwd antwoord te kunnen geven op de vraag of het nodig is om te moderniseren naar alternatieven.

\section{\IfLanguageName{dutch}{Onderzoeksdoelstelling}{Research objective}}
\label{sec:onderzoeksdoelstelling}

\subsection{\IfLanguageName{dutch}{Onderzoeksvraag 1: Zijn gedistribueerde systemen beter dan de moderne mainframe?}{Research question 1: Are distributed systems better than the modern mainframe?}}

De beoogde resultaten binnen deze eerste onderzoeksvraag, zal het resultaat zijn via een vergelijkende literatuurstudie. Er zal een uitgebreide studie gebeuren die de voor- en nadelen van gedistribueerde systemen in kaart zal brengen. De focus wordt gelegd op AWS en Google Cloud G4 platform. Eveneens gaat er gekeken worden welke nieuwigheden er op tafel liggen op vlak van Z technologieën. We gaan kijken naar de technische sterktes en zwaktes van de laatste nieuwe IBM mainframe.

\subsection{\IfLanguageName{dutch}{Onderzoeksvraag 2: Hebben instanties met grote batchverwerking plannen om de mainframe te vervangen of moderniseren naar gedistribueerde systemen?}{Research question 2: Do organizations with big batch processing plans to replace the mainframe with distributed systems?}}

Doormiddel van een enquête zal er in kaart gebracht worden of bedrijven met grote batchverwerking plannen hebben om te mainframe te moderniseren naar een nieuwe technologie. De resultaten die we hier willen uit afleiden is het aantal bedrijven op de totale deelname van de enquête die plannen hebben om te migreren. Indien ze geen plannen zouden hebben, willen we weten of ze het tekort aan experten ervaren en hiervoor al oplossingen hebben uitgewerkt. Anderzijds willen we te weten komen naar welke technologieën ze de overstap zouden maken als ze willen moderniseren. 

\subsection{\IfLanguageName{dutch}{Onderzoeksvraag 3: Wat is het kostenplaatje en complexiteit voor het migreren van een mainframe workload?}{Research question 3: What is the total cost and complexity for the migration of a mainframe workload?}}

Op basis van bevragingen zou er moeten in kaart gebracht worden wat het gemiddelde kostenplaatje is van een migratie. Eveneens de complexiteit van een migratie. Batchverwerking is een enorme workload en is vaak heel delicaat om aan te passen. Hoe zouden migrerende bedrijven dit aanpakken. Wat zijn hun plannen? Wat gaan ze doen met hun huidige mainframe experten?

\section{\IfLanguageName{dutch}{Opzet van deze bachelorproef}{Structure of this bachelor thesis}}
\label{sec:opzet-bachelorproef}

% Het is gebruikelijk aan het einde van de inleiding een overzicht te
% geven van de opbouw van de rest van de tekst. Deze sectie bevat al een aanzet
% die je kan aanvullen/aanpassen in functie van je eigen tekst.

De rest van deze bachelorproef is als volgt opgebouwd:

In Hoofdstuk~\ref{ch:stand-van-zaken} wordt een overzicht gegeven van de stand van zaken binnen het onderzoeksdomein, op basis van een literatuurstudie.

In Hoofdstuk~\ref{ch:methodologie} wordt de methodologie toegelicht en worden de gebruikte onderzoekstechnieken besproken om een antwoord te kunnen formuleren op de onderzoeksvragen.

% TODO: Vul hier aan voor je eigen hoofstukken, één of twee zinnen per hoofdstuk

In Hoofdstuk~\ref{ch:conclusie}, tenslotte, wordt de conclusie gegeven en een antwoord geformuleerd op de onderzoeksvragen. Daarbij wordt ook een aanzet gegeven voor toekomstig onderzoek binnen dit domein.