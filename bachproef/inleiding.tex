% !TeX spellcheck = de_DE
%%=============================================================================
%% Inleiding
%%=============================================================================

\chapter{\IfLanguageName{dutch}{Inleiding}{Introduction}}
\label{ch:inleiding}
Al 40  jaar lang beweren onderzoekers dat andere platformen het einde zullen betekenen voor de mainframe. Echter zijn er maar weinig bedrijven met enorme workloads die de stekker uit hun mainframe durven halen. Dagdagelijks zorgen 220 miljard lijnen COBOL of PL/1-code ervoor dat alle systemen binnen een organisatie blijven functioneren \autocite{Scannell2017}. De mainframe is verantwoordelijk voor miljarden transacties per dag zoals bijvoorbeeld een bestelling plaatsen op het internet aan de hand van bancontact of Visa. Mensen zijn zonder het zelf te beseffen dagelijks eindgebruiker van de mainframe, doch weten veel mensen niet wat een mainframe is \autocite{Scannell2017}. 

\section{\IfLanguageName{dutch}{De grote vraag: wat is een mainframe?}{The big question: what is a mainframe?}}

Een mainframe is simpelweg de grootste vorm van een server. Het is een machine die duizenden applicaties en I/O toestellen kan ondersteunen op een simultane manier.  Een mainframe doet dit op basis van de RAS-factoren. De reliability, availability en serviceability bij het verwerken van grote workloads zorgt ervoor dat miljarden gebruikers op een performante, veilige en betrouwbare manier interageren met hun data \autocite{Ebbers2022}. Met reliability wordt bedoeld dat een systeem uitgevoerd wordt waarvoor hij bedoeld is zonder optredende problemen gedurende een op voorhand gegeven tijdsinterval. Daarnaast is de availabilty de mogelijkheid van een systeem om op elk willekeurig moment in de tijd beschikbaar te zijn tijdens de looptijd van dat systeem. Ten slotte heeft de serviceability te maken met alle elementen van een systeemdesign \autocite{Johnson1988}.  Een mainframe moet gezien worden als de centrale datarepository. Het is een gecentraliseerd systeem dat gelinkt wordt aan minder krachtige machines zoals PC's.  Het voordeel van een gecentraliseerd systeem is dat de eindgebruikers hun businessdata maar één keer hoeven te updaten. Bij gedistribueerde systemen moet dat wel gebeuren en kan dit uiteindelijk zorgen voor corruptie van de data. Dit bewijst de eerste factor, reliability van de mainframe \autocite{Ebbers2022}. 

International Business Machines Corporation (IBM) is wereldwijd de bekendste mainframeproducent. Zij hebben een hele grote invloed op de evolutie van wat we vandaag begrijpen onder mainframe. In de jaren vijftig begon de ontwikkeling van Big Iron onder stoom te komen. IBM kwam er met de IBM 705 in 1954 waarop de IBM 1401 volgde in 1959. De doorbraak kwam echter pas in 1964. In dat jaar werd de computergeschiedenis helemaal omvergeblazen en was er een revolutie gestart waar we de technologie van vandaag aan te danken hebben. De IBM System/360 werd voor het eerst geïntroduceerd. Het was de eerste mainframe die hardware en software gestandaardiseerd had voor de eindgebruikers, die zowel wetenschappelijk als commercieel kon worden ingezet. Het was een kwestie van de juiste programma's te voorzien om specifieke doeleinden de bereiken. De keuze voor de naam heeft een achterliggende betekenis. Volgens \textcite{Ebbers2022} 360 in de naam staat namelijk voor het feit dat er een 360 graden aan mogelijke use cases bestaan waar deze mainframe kan bij ingezet worden.

Daarnaast beschrijft de architectuur in de computerwetenschappen  de organisatorische structuur van een systeem. Deze structuur kan opgedeeld worden in kleine schakelingen die op elkaar zijn afgestemd om adequaat en efficiënt te werken. Op vlak van architectuur zijn mainframes door de jaren heen het meest stabiel, compatibel en veilig gebleven doch enorm geëvolueerd  \autocite{Ebbers2022}. 

Laat ons even teruggaan naar de originele vraag wat een mainframe is. Die term kan volgens \textcite{Ebbers2022}  beschreven worden als een stijl van operatie, applicatie en voorzieningen van een besturingssysteem zoals IBM z/OS. Een mainframe is wat de bedrijfswereld ken gebruiken voor het hosten van commerciële databanken, transactieservers en applicaties. Dat laatste zijn menigmaal applicaties die een grotere veiligheid en beschikbaarheid eisen dan wat kleinschalige machines kunnen bieden. 

De eerste mainframesystemen waren enorme, metalen kasten. Zo\'n metalen kasten namen in de beginjaren van de mainframe ongeveer 200 tot 1000 vierkante meter in. Deze vereisten een hoge hoeveelheid aan elektriciteit en koeling. Een typisch datacenter had meerdere mainframe-installaties met een groot aantal I/0-toestellen die geconnecteerd werden aan deze mainframes. Rond 1990 werden mainframeprocessoren en hun I/O-toestellen fysiek kleiner. Toch nam de functionaliteit en capaciteit niet af, mainframes werden alsmaar krachtiger. De dag van vandaag hebben moderne mainframes de grootte van een koelkast. In sommige gevallen is het mogelijk om het besturingssysteem van een mainframe te draaien op een PC. Hierdoor is het mogelijk om een mainframe te emuleren. Deze emulators zijn praktisch voor het ontwikkelen en testen van businessapplicaties voor ze worden gepromoveerd naar de productieomgeving van een mainframe  \autocite{Ebbers2022}. 

\section{\IfLanguageName{dutch}{Wie gebruikt mainframecomputers?}{Who uses mainframe computers?}}

Iedereen is eindgebruiker van een mainframecomputer. Op een bepaald moment zijn we allemaal in contact gekomen met een mainframe. Bij het afhalen van cash geld aan een geldautomaat of bij het ontvangen van een terugbetaling van een ziekenfonds komen we vaak onbewust het mainframeplatform tegen \autocite{Ebbers2022}.Tegenwoordig spelen mainframes een centrale rol in de dagelijkse operaties van werelds grootste bedrijven. De mainframe is namelijk een gigantische pion in de wereld van e-business en e-commerce-omgevingen. Een mainframe biedt een ondersteunende kracht voor heel wat businessoperaties bij banken, verzekeringsinstellingen, overheidsinstellingen en groot aantal private bedrijven. Tot het midden van de jaren 90 waren mainframes het enige aanvaardbare middel om de vereisten op vlak van gegevensverwerking van een groot bedrijf aan te pakken. Deze vereisten bestonden dan uit grote en complexe batchjobs of online transacties zoals salaris- en grootboekverwerking.

Het vertrouwen in de mainframecomputer is te wijten aan de betrouwbaarheid en stabiliteit dat resulteert in een stabiele technologie. Een mainframecomputer blijft functioneel doorheen de jaren. De technologie is namelijk bestemd tegen anomalieën. Geen enkele andere computerarchitectuur heeft zo'n evolutionaire verbeteringen doorgemaakt met het behouden van de compatibiliteit van vorige versies. Door deze eigenschappen worden mainframes het meest gebruikt door instellingen en informatietechnologie organisaties om hun belangrijkste applicaties te hosten \autocite{Ebbers2022}. 

\newpage 

\section{\IfLanguageName{dutch}{Probleemstelling}{Problem Statement}}
\label{sec:probleemstelling}

Over de jaren heen heeft de mainframe heel wat concurrenten bijgekregen. Gedistribueerde technologieën zoals Amazon Web Services, Google Cloud Platform en Microservices zijn één van de bekendste tegenpolen van de mainframe die inzitten op migratie. Anderzijds heb je Intel of AMD servers die ook een waardige concurrent zijn van het mainframeplatform maar niet inzetten op migratiestrategieën voor mainframegebruikers. Eveneens de schaarste in mainframe-experten door het afnemende opleidingsaanbod is een probleem aan het worden. De bedrijven die een grote workload hebben waaronder CICS en Db2 worden meer en meer geconfronteerd met het in pensioen treden van hun mainframe-experten. Voor deze probleemstellingen zal een literatuurstudie en een vergelijkende studie uitgevoerd worden. Hieruit kan vervolgens geconcludeerd worden of de mainframe binnen grote organisaties en op vlak van innovatie een al dan niet een rooskleurige toekomst toegemoet gaat. Het kostenplaatje van een migratie en een mainframe is een belangrijke factor. Daarnaast speelt de complexiteit die een transitie van workloads met zich meebrengt eveneens een rol. 

\section{\IfLanguageName{dutch}{Onderzoeksvraag}{Research question}}
\label{sec:onderzoeksvraag}

De hoofdonderzoeksvraag van deze bachelorproef gaat over in welke mate gedistribueerde systemen de dag van vandaag vergelijkbaar zijn met een mainframe zoals de z16 mainframe van IBM. Daarnaast zal er in deze bachelorproef eveneens stilgestaan worden bij mogelijke plannen van instanties met grote workloads om de mainframe te vervangen of moderniseren aan de hand van gedistribueerde systemen zoals AWS. Hieruit wordt er meer informatie verkregen omtrent de voordelen en/of nadelen van het moderniseren van de mainframe. 

\section{\IfLanguageName{dutch}{Onderzoeksdoelstelling}{Research objective}}
\label{sec:onderzoeksdoelstelling}

\subsection{\IfLanguageName{dutch}{Onderzoeksvraag 1: in welke mate zijn gedistribueerde systemen de dag van vandaag vergelijkbaar met een mainframe zoals de z16 van IBM?}{Research question 1: to what extend are distributed systems comparable with a recent mainframe like the Z16 from IBM?}}

Om op deze onderzoeksvraag te antwoorden, wordt een vergelijkende (literatuur)studie uitgevoerd. Er zal een uitgebreide studie gebeuren die zowel de voor- als nadelen van gedistribueerde systemen in kaart zal brengen. De focus wordt gelegd op de gedistribueerde systemen AWS en Google Cloud platform. Eveneens worden mogelijke nieuwigheden op vlak van Z-technologieën onderzocht. Verder worden de technische sterktes en zwaktes van de IBM mainframe z16 bekeken.

\subsection{\IfLanguageName{dutch}{Onderzoeksvraag 2: hebben instanties met grote workloads plannen binnen nu en vijf jaar om de mainframe te vervangen of moderniseren naar gedistribueerde systemen?}{Research question 2: Do organizations with big batch processing have plans to replace the mainframe within today and five year by distributed systems?}}

Door middel van een enquête zal in kaart gebracht worden of bedrijven met grote batchverwerking reeds plannen hebben om de mainframe te moderniseren naar een nieuwe technologie. Aan de hand van de antwoorden op de enquête wordt afgeleid hoeveel deelnemende bedrijven effectief plannen om te migreren. Daarnaast willen we te weten komen naar welke technologieën de bedrijven de overstap zouden maken en welke redenen hen hiertoe drijven. 

Bovendien wordt er aan de hand van de enquête bevraagd of de bedrijven het tekort aan mainframe-experten ervaren en in welke mate dat een rol speelt binnen de migratieplannen. Eveneens wordt bevraagd welke mogelijke oplossingen het bedrijf voorziet om dat tekort te compenseren of verhelpen. 

\subsection{\IfLanguageName{dutch}{Onderzoeksvraag 3: wat is de complexiteit van het migreren van een mainframe workload de dag van vandaag en welke instellingen zijn hier in gespecialiseerd?}{Research question 3: What is the complexity for the migration of a mainframe workload today and which organizations are specialized in this matter?}}

Op basis van een literatuuronderzoek zal er op zoek gegaan worden naar instanties die gespecialiseerd zijn in het migreren van mainframeworkloads naar de cloud. Hier gaan we onderzoeken hoe complex dit is en wat volgens deze instanties zoals Astadia de reden is om dit te overwegen. Daarnaast nemen we een kijkje naar de complexiteit van een migratie of modernisatie van mainframe legacy. 

\newpage

\section{\IfLanguageName{dutch}{Opzet van deze bachelorproef}{Structure of this bachelor thesis}}
\label{sec:opzet-bachelorproef}

% Het is gebruikelijk aan het einde van de inleiding een overzicht te
% geven van de opbouw van de rest van de tekst. Deze sectie bevat al een aanzet
% die je kan aanvullen/aanpassen in functie van je eigen tekst.

De rest van deze bachelorproef is als volgt opgebouwd:

In Hoofdstuk~\ref{ch:stand-van-zaken} zal een literatuurstudie worden gedaan omtrent de tekorten aan expertise op vlak van mainframe, de nieuwe IBM Z16 mainframe, Google Cloud en AWS. Daarnaast een praktische uitwerking waaruit de complexiteit van een migratie naar AWS kan worden geconcludeerd. Ten slotte zal hier ook een toelichting worden gegeven over de resultaten van de bevraging bij bedrijven, studenten en IT-professionals. 

In Hoofdstuk~\ref{ch:methodologie} wordt de methodologie toegelicht die werd gebruikt voor het voeren van het betreffende onderzoek. Aan de hand van een literatuurstudie, praktische uitwerking en bevraging werden resultaten bereikt die leiden tot het antwoord op de vooropgestelde onderzoeksvragen. De manier waarop dit werd gedaan staat beschreven in de methodologie.

% TODO: Vul hier aan voor je eigen hoofstukken, één of twee zinnen per hoofdstuk

In Hoofdstuk~\ref{ch:conclusie}, tenslotte, wordt de conclusie gegeven en een antwoord geformuleerd op de onderzoeksvragen. Daarbij wordt ook een aanzet gegeven voor toekomstig onderzoek binnen dit domein.