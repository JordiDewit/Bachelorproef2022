\chapter{\IfLanguageName{dutch}{Stand van zaken}{State of the art}}
\label{ch:stand-van-zaken}

% Tip: Begin elk hoofdstuk met een paragraaf inleiding die beschrijft hoe
% dit hoofdstuk past binnen het geheel van de bachelorproef. Geef in het
% bijzonder aan wat de link is met het vorige en volgende hoofdstuk.

% Pas na deze inleidende paragraaf komt de eerste sectiehoofding.

In deze stand van zaken zal verduidelijkt worden wat de hedendaagse problemen en bezorgdheden zijn in de wereld van de mainframe. Aan de hand van een literatuurstudie worden de problemen in kaart gebracht. 

\section{\IfLanguageName{dutch}{Het verdwijnen van expertise}{Lostage from expertise}}
\label{sec:Verdwijnen van expertise}

Veel informatietechnolgie specialisten, zoals Stewart Alsop voorspelden dat de stekker uit de mainframe zou worden gehaald aan het einde van de jaren 90 \autocite{McCracken2012}. Volgens het onderzoek van \textcite{Waites2013} zal de mainframe nog niet verdwijnen. Daarentegen verdwijnt de expertise wel. Mainframe-applicaties die kritisch zijn voor de werking van bedrijfsprocessen zorgen voor een angst dat de complexe expertise voor het ontwikkelen en onderhouden van deze applicaties aan het verdwijnen is. De mainframe-ontwikkelaars die geboren zijn tussen 1945 en 1964 zijn al gepensioneerd of gaan op pensioen. Dayton Semerjian, een senior vice-president bij CA Technologies zei ''The average age of mainframe workers is 55 to 60''. CA Technologies is de op IBM na grootste mainframeproducent. Veel organisaties zien outsourcing van hun workloads als een oplossing voor de tekorten van experten. Dit is een manier waarop ze een compensatie proberen te creeëren om hun mainframe workloads te behouden, doch tegelijkertijd te onderhouden en verbeteren. Het problematisch aspect hierbij is dat het heel tijdrovend en vermoeiend is voor een organisatie om de benodigde kennis door te geven. Bij mainframe-ontwikkeling is het vaak zo dat je heel wat kan doen met technische kennis an sich, maar kennis over de organisatie is van even groot belang.  

Sinds 1960, het jaar dat dr. Grace Hopper en haar collega's COBOL hebben gecreëerd. Deze programmeertaal is naar schatting gegroeid van 150 tot 250 miljard lijnen applicatiecode, waarvan acht tot negen miljoen nieuwe lijnen per jaar worden bijgeschreven. Het is nog steeds de populairste programmeertaal in de mainframewereld. Anderzijds met de ontwikkeling van andere programmeertalen, wordt COBOL afgeschreven als 'out-of-date' en 'dying'. Elk jaar sinds 1970 werd het uitsterven van COBOL een volkomen zekerheid. Als een organisatie een mainframe heeft, zijn de kansen zeer groot dat al hun applicaties geschreven zijn in COBOL. Programmeurs die in de jaren 70 en 80 in de technologiesector instapten, was het heel waarschijnlijk dat zij een COBOL programmeur werden. Daarom is het duidelijk dat de leeftijd van mainframe-experten verschillend is van de leeftijddistributies die in de UNIX en Windows omgevingen plaatsvindt \autocite{McGirr}. Volgens studies van Meta Group zijn 60\% van de mensen die werken op een mainframeomgeving  50 jaar of ouder in vergelijking met de 8\% in een UNIX- of Windowsomgeving. Slechts vijf percent zijn onder de 30 jaar. Het is hieruit duidelijk te concluderen dat veel mainframe-experten de pensioenleeftijd aan het naderen zijn. 