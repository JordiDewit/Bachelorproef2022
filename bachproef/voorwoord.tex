%%=============================================================================
%% Voorwoord
%%=============================================================================

\chapter*{\IfLanguageName{dutch}{Woord vooraf}{Preface}}
\label{ch:voorwoord}

%% TODO:
%% Het voorwoord is het enige deel van de bachelorproef waar je vanuit je
%% eigen standpunt (``ik-vorm'') mag schrijven. Je kan hier bv. motiveren
%% waarom jij het onderwerp wil bespreken.
%% Vergeet ook niet te bedanken wie je geholpen/gesteund/... heeft

Als keuzepakket in mijn laatste jaar van de professionale bachelor toegepaste informatica, koos ik  voor de afstudeerrrichting mainframe-expert. In dat keuzepakket heb ik de wondere wereld van de mainframecomputer mogen ontdekken. Doordat deze kleine wereld mij zich in zijn armen heeft gesloten, vond ik dat het tijd werd om te onderzoeken hoe de toekomst van deze machine eruitziet. Tijdens het schrijven van deze bachelorproef liep ik stage als mainframe-ontwikkelaar bij Het Nationaal Verbond Van Socialistische Mutualiteiten. Hierdoor ging ik dagelijks met deze technologie aan te slag. Ik heb geen enkel moment spijt gehad dat ik voor deze afstudeerrichting heb gekozen. Mijn wens is om studenten of IT-professionals te overtuigen om net zoals ik mainframe-expert te worden. 

Allereerst wil ik mijn promotor Chantal Teerlinck, opleidingshoofd van de opleiding toegepaste informatica aan de Hogeschool Gent, bedanken voor de feedback en de goede begeleiding doorheen het proces. Daarnaast wil ik mijn co-promotor Jan Cannaerts,  systeemadministrator op mainframe, bedanken om mij inhoudelijk bij te sturen tijdens de ontwikkeling van dit onderzoek. Ik wil eveneens mijn vriendin Yna Bauwens bedanken om mijn steun en toeverlaat te zijn gedurende het schrijven van deze bachelorproef. Ten slotte wil ik alle participanten van de enquêtes bedanken, de organisatoren van de GSE te Almere en het IBM Z technologies team. 

Ik wens jullie veel plezier bij het lezen van mijn bachelorproef en hopelijk overtuig ik jullie om in de wereld van de mainframe te stappen. 

Jordi Dewit \newline
Asse, juni 2022