%%=============================================================================
%% Samenvatting
%%=============================================================================

% TODO: De "abstract" of samenvatting is een kernachtige (~ 1 blz. voor een
% thesis) synthese van het document.
%
% Deze aspecten moeten zeker aan bod komen:
% - Context: waarom is dit werk belangrijk?
% - Nood: waarom moest dit onderzocht worden?
% - Taak: wat heb je precies gedaan?
% - Object: wat staat in dit document geschreven?
% - Resultaat: wat was het resultaat?
% - Conclusie: wat is/zijn de belangrijkste conclusie(s)?
% - Perspectief: blijven er nog vragen open die in de toekomst nog kunnen
%    onderzocht worden? Wat is een mogelijk vervolg voor jouw onderzoek?
%
% LET OP! Een samenvatting is GEEN voorwoord!

%%---------- Nederlandse samenvatting -----------------------------------------
%
% TODO: Als je je bachelorproef in het Engels schrijft, moet je eerst een
% Nederlandse samenvatting invoegen. Haal daarvoor onderstaande code uit
% commentaar.
% Wie zijn bachelorproef in het Nederlands schrijft, kan dit negeren, de inhoud
% wordt niet in het document ingevoegd.

\IfLanguageName{english}{%
\selectlanguage{dutch}
\chapter*{Samenvatting}
\lipsum[1-4]
\selectlanguage{english}
}{}

%%---------- Samenvatting -----------------------------------------------------
% De samenvatting in de hoofdtaal van het document

\chapter*{\IfLanguageName{dutch}{Samenvatting}{Abstract}}

Uit literatuur en diverse onderzoeken blijkt dat de mainframe populariteit aan het verliezen is zowel in het opleidingsaanbod als in instellingen met grote workloads. Zijn deze alternatieve systemen zoals de cloud dan werkelijk de oplossing? Daarom werd het tijd dat er een onderzoek werd uitgevoerd om het uitsterven van de mainframe in vraag te trekken. In dit onderzoek werd onderzocht wat het verschil is tussen de mainframecomputer en gedistribueerde systemen, de complexiteit van een migratie en het hedendaagse tekort aan mainframe-expertise. Het onderzoek gebeurde door middel van een literatuur- en vergelijkende studie, een bevraging en een praktische uitwerking. De verschillen tussen de onderzochte platformen werden beschreven via een vergelijkende literatuurstudie. Daarnaast werd het expertisetekort onderzocht aan de hand van een literatuurstudie en een bevraging. Tot slot werd de complexiteit in kaart gebracht met behulp van een praktische uitwerking om een mainframe COBOL-applicatie uitvoerbaar te maken op het Amazon Web Services platform. Om instellingen met mainframeworkloads te helpen met hun keuze om te migreren kan dit onderzoek de doorslaggevende factor zijn. De resultaten zijn op vlak van complexiteit en future proof-technologie in het voordeel van de mainframe. Deze computer biedt op vlak van de RAS-factoren en security, features die voorbereid zijn op toekomstige problemen zoals quantum computed hacking. Echter is het wel zo dat de resultaten van het expertisetekort problematisch zijn. Het vraag- en aanbodmodel voor het vinden van mainframe-experten is een factor die instellingen aanzetten om een migratie uit te voeren. Veel te weinig academische instellingen spannen zich in om deze problemen van de baan te helpen. Door middel van de praktische uitwerkingen werd in het onderzoek geconcludeerd dat de complexiteit van een migratie een kritische factor is om het slagen van migraties naar de cloud in twijfel te trekken. Als gevolg hiervan geeft dit onderzoek aanzet om te onderzoeken wat de slagingskansen zijn voor een succesvolle migratie van real life mainframeworkloads. 