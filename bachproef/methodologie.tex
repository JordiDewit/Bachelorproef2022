%%=============================================================================
%% Methodologie
%%=============================================================================

\chapter{\IfLanguageName{dutch}{Methodologie}{Methodology}}
\label{ch:methodologie}

%% TODO: Hoe ben je te werk gegaan? Verdeel je onderzoek in grote fasen, en
%% licht in elke fase toe welke stappen je gevolgd hebt. Verantwoord waarom je
%% op deze manier te werk gegaan bent. Je moet kunnen aantonen dat je de best
%% mogelijke manier toegepast hebt om een antwoord te vinden op de
%% onderzoeksvraag.

%\lipsum[21-25]

\section{\IfLanguageName{dutch}{De literatuur- en vergelijkende studie}{Literature and comparitive study}}
\label{sec:De literatuur- en vergelijkende studie}

Ik ben het onderzoek gestart door op zoek te gaan naar bevestigingen omtrent de vooroordelen rond vergrijzing en afname van de mainframe-expertise. Hiervoor heb ik een combinatie willen implementeren van zowel oudere bronnen en recente bronnen. Daarnaast heb ik voor dit onderwerp een bevraging opgesteld waaruit moet kunnen geconcludeerd worden dat hedendaagse bedrijven werkelijk met deze problemen kampen. 

Vervolgens heb ik de laatste nieuwe IBM Z16 mainframe onderzocht. De bedoeling was dat ik de meest interessante nieuwigheden en innovaties kon schetsen voor de lezers van dit onderzoek. Voor die reden heb ik specifiek gezocht naar artikels waar journalisten de belangrijkste veranderingen aankaarten. Doormiddel van een excursie naar het IBM center in Montpellier wist ik al een tijdje voor de introductie dat er een nieuwe mainframe op komst was en daarom vond ik het zeker waard om deze nieuwe machine te vergelijken met gedistribueerde alternatieven.


\section{\IfLanguageName{dutch}{De bevraging}{The survey}}
\label{sec:De bevraging}


\section{\IfLanguageName{dutch}{De praktische uitwerking}{The practical implementation}}
\label{sec:De praktische uitwerking}
