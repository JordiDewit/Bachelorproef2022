%%=============================================================================
%% Methodologie
%%=============================================================================

\chapter{\IfLanguageName{dutch}{Methodologie}{Methodology}}
\label{ch:methodologie}

%% TODO: Hoe ben je te werk gegaan? Verdeel je onderzoek in grote fasen, en
%% licht in elke fase toe welke stappen je gevolgd hebt. Verantwoord waarom je
%% op deze manier te werk gegaan bent. Je moet kunnen aantonen dat je de best
%% mogelijke manier toegepast hebt om een antwoord te vinden op de
%% onderzoeksvraag.

%\lipsum[21-25]

\section{\IfLanguageName{dutch}{De literatuur- en vergelijkende studie}{Literature and comparitive study}}
\label{sec:De literatuur- en vergelijkende studie}

Het onderzoek werd gestart door op zoek te gaan naar bevestigingen omtrent de vooroordelen rond vergrijzing en de afname van mainframe-expertise. Hiervoor werd een combinatie geïmplementeerd van zowel oudere bronnen als recente bronnen. Daarnaast werd voor dit onderwerp een bevraging opgesteld waaruit moet kunnen geconcludeerd worden dat hedendaagse bedrijven werkelijk met deze problemen kampen. 

Vervolgens werd de laatste nieuwe IBM z16 mainframe onderzocht. De bedoeling hiervan was dat de meest interessante nieuwigheden en innovaties konden geschetst worden voor dit onderzoek. Voor deze reden is uitsluitend op zoek gegaan naar artikels waar journalisten de belangrijkste nieuwe features aankaarten. Door middel van een excursie naar het IBM service center in Montpellier was er al een tijdje voor de introductie geweten dat er een nieuwe mainframe op komst was. Daarom was het zeker waard om deze nieuwe machine te vergelijken met gedistribueerde alternatieven. Er werd hierbij vermeden om informatie te halen uit bronnen die in directe verbinding staan met het sales team IBM. Dat zou de uitspraken over de IBM z16 te subjectief maken. 

Om onderzoek te doen naar een instelling die gespecialiseerd is in het moderniseren van mainframe workloads, werd gekozen voor Astadia. Dat bedrijf publiceerde een whitepaper waar ze dieper op alle aspecten rond een migratie naar Google Cloud ingaan. In deze whitepaper gaven zij hun argumenten aan waarom Google Cloud beter zou zijn dan een traditionele mainframe voor het verwerken van grote workloads. 


\section{\IfLanguageName{dutch}{De bevraging}{The survey}}
\label{sec:De bevraging}

Voor de bevraging werd Google Forms gebruikt. Dat was de ideale tool om op een professionele manier een enquête op te stellen waaruit we de gewenste informatie konden halen. Om antwoorden te verkrijgen van de juiste mensen, werd er nagedacht over het juiste platform voor het delen van deze enquête. Hiervoor werd LinkedIn gehanteerd. Daarnaast werd de enquête via mail gestuurd naar verschillende bedrijven. 

In het onderzoek zijn twee enquêtes opgesteld voor verschillende doelgroepen. De ene doelgroep zijn organisaties die al dan niet workloads hebben die op een mainframe verwerkt worden. Het was de bedoeling dat deze doelgroep een antwoorden gaven op de vragen rond gebrek aan expertise en migratieplannen. De andere doelgroep waren studenten computerwetenschappen en IT-professionals. Zij dienden antwoorden te geven over hoe zij staan tegenover mainframetechnologie en of zij geïnteresseerd zijn om hier meer over te leren. Hieruit kon in dit onderzoek een conclusie worden getrokken over waar de interesses liggen van de meeste IT-professionals en studenten.

Hieronder bevinden zich de vragen die werden opgesteld voor de organisaties met mainframeworkloads:
 \begin{itemize}
    \item Wat is de naam van uw organisatie?
    \item In welk sector is uw organisatie actief?
    \item Maakt uw organisatie gebruik van een mainframe?
    \item Hoeveel werknemers ontwikkelen, onderhouden of beheren software op een mainframe?
    \item Wat is de gemiddelde leeftijd van het mainframepersoneel binnen uw organisatie?
    \item Zijn er signalen die weergeven dat er een gebrek is aan nieuwe mainframe-experten?
    \item Wat is de visie van uw organisatie op de expertise rond mainframe?
    \item Zijn er initiatieven genomen om interne opleidingen aan te bieden rond mainframetechnologie?
    \item Waarvoor maakt uw organisatie gebruik van een mainframe?
    \item Wat zijn voor uw organisatie de voordelen van een mainframe ten opzichte van andere platformen?
    \item Wat zijn de nadelen van het mainframeplatform voor uw organisatie?
    \item Zijn er migratieplannen naar andere platformen? Waarom wel/niet?
    \item Indien er migratieplannen aanwezig zijn, naar welke platformen zouden worden gemigreerd?
    \item Wat is de gemiddelde jaarlijkse kost van een mainframe voor uw organisatie?
    \item Wat is de producent van de mainframe die uw organisatie bezit of gebruikt?
    \item Zijn er plannen om nieuwe technologieën te implementeren op het mainframeplatform van uw organisaties?
\end{itemize}

\newpage

Vervolgens bevinden zich hieronder de vragen die werden opgesteld voor de organisaties zonder mainframeworkloads:
 \begin{itemize}
    \item Waarom gebruikt uw organisatie geen mainframe?
    \item Welk alternatief platform gebruikt uw organisatie?
    \item Wat is de gemiddelde jaarlijkse kost van dat alternatief platform?
\end{itemize}

Ten slotte bevinden zich de vragen voor studenten en IT-professionals hieronder:
 \begin{itemize}
    \item Wat is uw functie binnen IT? (Student, ontwikkelaar, analist, ...)
    \item Weet u wat een mainframe is?
    \item Bent u al in contact gekomen met een mainframe op het werk of op school?
    \item Heeft u ervaring met COBOL of PL/1?
    \item Zou u geïnteresseerd zijn om te ontwikkelen op een mainframeplatform?
    \item Indien u student bent, zou u interesse hebben om voor mainframe-expert te studeren?
    \item Waarom zou u wel/niet kiezen om mainframe-expert te worden?
\end{itemize}



\section{\IfLanguageName{dutch}{De praktische uitwerking}{The practical implementation}}
\label{sec:De praktische uitwerking}

Voor de praktische uitwerking is er op zoek gegaan naar een manier om aan te tonen wat de complexiteit is van een migratie. Om dit te schetsen werd een simpel ``Hello world `` COBOL applicatie gemigreerd naar AWS. Omtrent dat klein project werd interessante informatie gevonden over het gebruik van een AWS Lambda functie en een publicatie met een gids. Deze gids werd vervolledigd om een kleinschalige migratie naar een gedistribueerde vorm van informatica te ervaren. Daaruit was het mogelijk om een inschatting te maken van de complexiteit van een migratie van een real life workload met afhankelijkheden en duizenden gebruikers.
