%==============================================================================
% Sjabloon onderzoeksvoorstel bachelorproef
%==============================================================================
% Gebaseerd op LaTeX-sjabloon ‘Stylish Article’ (zie voorstel.cls)
% Auteur: Jens Buysse, Bert Van Vreckem
%
% Compileren in TeXstudio:
%
% - Zorg dat Biber de bibliografie compileert (en niet Biblatex)
%   Options > Configure > Build > Default Bibliography Tool: "txs:///biber"
% - F5 om te compileren en het resultaat te bekijken.
% - Als de bibliografie niet zichtbaar is, probeer dan F5 - F8 - F5
%   Met F8 compileer je de bibliografie apart.
%
% Als je JabRef gebruikt voor het bijhouden van de bibliografie, zorg dan
% dat je in ``biblatex''-modus opslaat: File > Switch to BibLaTeX mode.

\documentclass{voorstel}
\usepackage{lipsum}
\usepackage{graphicx}
\usepackage{caption}
\usepackage{subcaption}

%------------------------------------------------------------------------------
% Metadata over het voorstel
%------------------------------------------------------------------------------

%---------- Titel & auteur ----------------------------------------------------

% TODO: geef werktitel van je eigen voorstel op
\PaperTitle{Moderniseren van `Legacy' en 'Big Iron': Literatuuronderzoek en vergelijkende studie}
\PaperType{Onderzoeksvoorstel Bachelorproef 2021-2022} % Type document

% TODO: vul je eigen naam in als auteur, geef ook je emailadres mee!
\Authors{Jordi Dewit\textsuperscript{1}} % Authors
\CoPromotor{Jan Cannaerts\textsuperscript{2} (Nationaal Verbond Van Socialistische Mutualiteiten)}
\affiliation{\textbf{Contact:}
  \textsuperscript{1} \href{mailto:jordi.dewit@student.hogent.be}{jordi.dewit@student.hogent.be};
  \textsuperscript{2} \href{mailto:jan.cannaerts@socmut.be}{jan.cannaerts@socmut.be};
}

%---------- Abstract ----------------------------------------------------------
\Abstract{In deze bachelorproef zal er een onderzoek gedaan worden naar de alternatieven voor een mainframe, ook wel Big Iron \autocite{Michael2021} genoemd en de technologieën die het biedt om grote workloads te verwerken. Alternatieven zoals Google Cloud G4 Platform en AWS zullen onderzocht worden om na te gaan of deze alternatieven een waardige opvolger zijn van de mainframe. Ook gaat er onderzocht worden wat de complexiteit is om over te stappen naar een ander platform. Welke uitdagingen liggen er op deze weg? Wat is het kostenplaatje van een migratie? Ten slotte gaat er ook gekeken worden naar welke nieuwe technologieën de mainframe ons in deze huidige maatschappij nog te bieden heeft. Dat gaat gebeuren via literatuurstudies, interviews en een project. Aan de hand van deze methodologieën zal er een antwoord kunnen gegeven worden op de vraag of mainframes nog relevant zijn of er toch beter gebruik wordt van de cloud voor onze dagdagelijkse functionaliteit zoals banktransacties. Mijn verwachtingen zijn zeer positief als het gaat om de toekomst van mainframes. Echter is dat natuurlijk heel subjectief. Daarom zal dit onderzoek een eerder objectief antwoord geven op de vraag of er echt moet weg gestapt worden van Big Iron.
}

%---------- Onderzoeksdomein en sleutelwoorden --------------------------------
% TODO: Sleutelwoorden:
%
% Het eerste sleutelwoord beschrijft het onderzoeksdomein. Je kan kiezen uit
% deze lijst:
%
% - Mobiele applicatieontwikkelings
% - Webapplicatieontwikkeling
% - Applicatieontwikkeling (andere)
% - Systeembeheer
% - Netwerkbeheer
% - Mainframe
% - E-business
% - Databanken en big data
% - Machineleertechnieken en kunstmatige intelligentie
% - Andere (specifieer)
%
% De andere sleutelwoorden zijn vrij te kiezen

\Keywords{Onderzoeksdomein. Mainframe --- Cloud --- Toekomstvisie --- IBM Z Technologie} % Keywords
\newcommand{\keywordname}{Sleutelwoorden} % Defines the keywords heading name

%---------- Titel, inhoud -----------------------------------------------------

\begin{document}

\flushbottom % Makes all text pages the same height
\maketitle % Print the title and abstract box
\tableofcontents % Print the contents section
\thispagestyle{empty} % Removes page numbering from the first page

%------------------------------------------------------------------------------
% Hoofdtekst
%------------------------------------------------------------------------------

% De hoofdtekst van het voorstel zit in een apart bestand, zodat het makkelijk
% kan opgenomen worden in de bijlagen van de bachelorproef zelf.
%---------- Inleiding ---------------------------------------------------------

\section{Introductie} % The \section*{} command stops section numbering
\label{sec:introductie}
Big Iron of mainframes zijn zeer grote computers die gebruikt worden door vaak grote instituten en bedrijven. Zij gebruiken deze technologie om veel en meerdere gebruikers tegelijk alles te bieden wat ze nodig hebben op vlak van data. Mainframes zijn in staat om enorme hoeveelheden data te verwerken op een snelle en consistente manier \autocite{Ebbers2016}. Dit soort legacy systemen gaan al mee sinds de jaren 50 en worden de dag van vandaag nog steeds gebruikt. Daarom is er een onderzoek nodig naar het wegstappen van deze legacy systemen en het moderniseren naar meer hedendaagse technologieën zoals de cloud. Het is belangrijk om een beeld te krijgen van hoe realistisch dit is. Als gevolg hiervan zal eventueel ook de nood aan mainframetechnologie binnen de moderne maatschappij duidelijk worden. Mijn verwachtingen zijn dat het heel complex en duur is om hiervan weg te stappen. Er draaien dagelijks miljoenen batchprogramma’s en transacties om zaken te verrichten die we vanzelfsprekend vinden. Er zijn heel wat problemen dat deze transformatie met zich meebrengt  \autocite{Long2018}.
\begin{itemize}
    \item Mainframe afhankelijke applicaties
    \item Code refactoring
    \item Workload complexiteit
    \item Kosten
    \item Risico's
    \item Databank specifieke problemen zoals gedeelde opslag
\end{itemize}
Deze problemen maken het een complex gegeven om zomaar van de ene dag op de andere een volledige mainframe-infrastructuur om te gooien naar gedistribueerde systemen. Hoe moeilijk dit is, zal afgeleid kunnen worden uit dit onderzoek. De volgende onderzoeksvragen zijn hiervoor belangrijk:
\begin{itemize}
    \item Welke alternatieven bestaan er en wat zijn de voor- en nadelen tegenover een mainframe?
    \item Welke technologieën worden er op dit moment nog ontwikkeld?
    \item Zullen we ooit kunnen wegstappen van de mainframe?
\end{itemize}
\textbf{De algemene probleemstelling:} wegstappen van legacy systemen is complex en duur \newline
\textbf{Doelstelling van het onderzoek:} onderzoeken wat de toekomst inhoudt voor mainframes \newline
\textbf{De hoofdvraag:} zijn alternatieven zoals Google Cloud G4 Platform goed genoeg om mainframes effectief te vervangen?

%---------- Stand van zaken ---------------------------------------------------

\section{Sneak peek}
\label{sec:state-of-the-art}
In 1991 had technologie journalist Stewart Alsop voorspeld dat op 15 maart 1996, de laatste mainframe uitgeschakeld zou worden. Gedistribueerde systemen zouden de toekomst zijn, doch ondanks deze voorspellingen bestaat de mainframe in 2022 nog steeds \autocite{Barnett2005}. IBM heeft nu met de Z-series al een initiatief genomen om de mainframe in te zetten als essentieel deel van een hybrid cloud strategie. IBM blijft steeds innoveren op vlak van prijs en innovatie \autocite{Smolaks2021}. Echter ondanks deze innovaties zijn de kosten van mogelijke alternatieven alsmaar interessanter aan het worden voor bedrijven. Als gevolg hiervan is het voor vele een goede optie om weg te migreren van hun mainframe-infrastructuur. Er zijn daarentegen ook al verschillende bedrijven zoals Micro Focus en Sun die “Middleware” aanbieden dat dient als een emulator voor een mainframe-omgeving. Dat zorgt ervoor dat klanten hun mainframespecifieke programma’s kunnen migreren zonder het herschrijven van de code. Dit wordt ook wel “drag and drop migration” genoemd. Het is altijd een hoog risico om legacy applicaties te gaan vernieuwen. Het doel van deze inspanningen komt altijd neer op het verlagen van de hardware kosten die legacy systemen met zich meebrengen. Vaak zijn de kosten daarentegen van het vernieuwen van de applicaties hoger dan wat men zou uitsparen met goedkopere hardware platformen \autocite{Bingell2014}. Mainframes brengen jammer genoeg heel wat nadelen met zich mee die de drang naar migratie alleen maar versterken. Een daarvan is het tekort aan experts. Vele onder hen zijn de pensioenleeftijd aan het naderen en heel wat bedrijven hebben nog geen concrete beleidsplannen klaar om dit probleem aan te pakken volgens het onderzoek van Meta Group in 2004 \autocite{Bakker2006}. Werknemers die gestart zijn in 1970 en 1980 zijn nu het einde van hun carrière aan het naderen. Dit geeft veel bedrijven nog een extra reden om weg te migreren van hun ‘Big Iron’ infrastructuur. Alsook bieden weinig scholen nog een opleiding aan voor mainframe expert. Hogeschool Gent is de enige hogeschool in België waar je kan afstuderen als mainframe expert.

 

%---------- Methodologie ------------------------------------------------------
\section{Methodologie} 
\label{sec:methodologie}
De verzameling van data zal gebeuren door: \newline
Literatuur onderzoek:
\begin{itemize}
    \item AWS, Google Cloud G4 platform 
    \item Technische sterktes en zwaktes van mainframe en cloud (vergelijkende studie)
    \item Geschiedenis van mainframes
    \item Noodzaak in huidige maatschappij a.d.h.v. mainframe specifieke use cases
\end{itemize}
Interviews binnen bedrijven zoals IBM, Nationaal Verbond Van Socialistische Mutualiteiten, Arselor Mittal, Colruyt.. Deze interviews zullen voornamelijk gaan over:
\begin{itemize}
     \item Beleidsplannen omtrent vergrijzing en tekort aan specialisten
     \item Migratieplannen
     \item Financiële situatie
     \item IBM Modernisatie en nieuwe technologieën 
     \item Redenen om te migreren naar alternatief
\end{itemize}
Het volgen van introducties van nieuwe technologie, hetzij ten voordele of ten nadele van de mainframe, zal ook gebeuren in het onderzoek. Hieruit kan een vergelijkende studie gedaan worden. Ook zal er meegewerkt worden aan een project binnen “Nationaal Verbond Van Socialistische Mutualiteiten”. Dit project omvat de modernisering van programma’s die maandelijkse boekhoudafsluitingen verwerken. Deze programma’s moeten momenteel dagelijks de boekafsluitingen voorzien. Aan de hand van dit project zal er onderzocht worden of dat ook efficiënt en veilig zou kunnen in een cloud omgeving. \newpage
%---------- Verwachte resultaten ----------------------------------------------
\section{Verwachte resultaten}
\label{sec:verwachte_resultaten}
De eerste voorbeeldgrafiek is een vergelijking van de gemiddelde leeftijd van IT'ers binnen verschillende domeinen. Zie figuur \ref{fig:chart1}. \newline
Op de tweede voorbeeldgrafiek wordt er onderzocht wat de verhouding is tussen afstuderende mainframe experts, diegene die de pensioenleeftijd naderen en die ertussenin. Zie figuur \ref{fig:chart2}. \newline
Op de derde voorbeeldgrafiek wordt er gekeken naar te onderzoeken kostenfactoren die mainframes en gedistribueerde systemen met zich meebrengen. Zie figuur \ref{fig:chart3}. \newline
Op de vierde voorbeeldgrafiek worden de MIPS (Millions of instructions per second) vergeleken tussen bepaalde mainframe modellen en gedistribueerde systemen. Zie figuur \ref{fig:chart4}. \newline


%---------- Verwachte conclusies ----------------------------------------------
\section{Verwachte conclusies}
\label{sec:verwachte_conclusies}
Mijn verwachtingen uit het onderzoek zijn in het voordeel van een mainframe. Volgens mij gaat het nog veel jaren duren vooraleer de mainframe effectief volledig zal vervangen worden door gedistribueerde systemen zoals AWS. Daarentegen heb ik wel slechte verwachtingen als het gaat over de populariteit en het studieaanbod voor het aanleren van mainframespecifieke skills. Als we deze problemen goed kunnen aanpakken in de toekomst en werken aan de gebruiksvriendelijkheid van een mainframe, dan ziet de toekomst van de mainframe er volgens mij zeer rooskleurig uit. Ten slotte verwacht ik te concluderen uit het onderzoek dat het migreren naar een cloud omgeving heel wat problemen met zich mee zullen brengen, zowel op vlak van veiligheid, stabiliteit als kostefficiëntie. 
\begin{figure}
    \centering
    \begin{subfigure}{0.3\textwidth}
    \centering
    \includegraphics[width=\textwidth]{../../../Chart1}
    \caption{}
    \label{fig:chart1}
    \end{subfigure}
    \hfill
    \begin{subfigure}{0.3\textwidth}
        \centering
        \includegraphics[width=\textwidth]{../../../Chart2}
        \caption{}
        \label{fig:chart2}
    \end{subfigure}
    \hfill
    \begin{subfigure}{0.3\textwidth}
        \centering
        \includegraphics[width=\textwidth]{../../../Chart3}
        \caption{}
        \label{fig:chart3}
    \end{subfigure}
    \hfill
    \begin{subfigure}{0.3\textwidth}
        \centering
        \includegraphics[width=\textwidth]{../../../Chart4}
        \caption{}
        \label{fig:chart4}
    \end{subfigure}
\caption{Voorbeeld diagrammen}
\end{figure}






%------------------------------------------------------------------------------
% Referentielijst
%------------------------------------------------------------------------------
% TODO: de gerefereerde werken moeten in BibTeX-bestand ``voorstel.bib''
% voorkomen. Gebruik JabRef om je bibliografie bij te houden en vergeet niet
% om compatibiliteit met Biber/BibLaTeX aan te zetten (File > Switch to
% BibLaTeX mode)

\phantomsection
\printbibliography[heading=bibintoc]

\end{document}
