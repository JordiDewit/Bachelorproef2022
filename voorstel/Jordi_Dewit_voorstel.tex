%==============================================================================
% Sjabloon onderzoeksvoorstel bachelorproef
%==============================================================================
% Gebaseerd op LaTeX-sjabloon ‘Stylish Article’ (zie voorstel.cls)
% Auteur: Jens Buysse, Bert Van Vreckem
%
% Compileren in TeXstudio:
%
% - Zorg dat Biber de bibliografie compileert (en niet Biblatex)
%   Options > Configure > Build > Default Bibliography Tool: "txs:///biber"
% - F5 om te compileren en het resultaat te bekijken.
% - Als de bibliografie niet zichtbaar is, probeer dan F5 - F8 - F5
%   Met F8 compileer je de bibliografie apart.
%
% Als je JabRef gebruikt voor het bijhouden van de bibliografie, zorg dan
% dat je in ``biblatex''-modus opslaat: File > Switch to BibLaTeX mode.

\documentclass{voorstel}

\usepackage{lipsum}

%------------------------------------------------------------------------------
% Metadata over het voorstel
%------------------------------------------------------------------------------

%---------- Titel & auteur ----------------------------------------------------

% TODO: geef werktitel van je eigen voorstel op
\PaperTitle{De toekomst van 'Big Iron'}
\PaperType{Onderzoeksvoorstel Bachelorproef 2021-2022} % Type document

% TODO: vul je eigen naam in als auteur, geef ook je emailadres mee!
\Authors{Jordi Dewit\textsuperscript{1}} % Authors
\CoPromotor{Jan Cannaerts\textsuperscript{2} (Nationaal Verbond Van Socialistische Mutualiteiten)}
\affiliation{\textbf{Contact:}
  \textsuperscript{1} \href{mailto:jordi.dewit@student.hogent.be}{jordi.dewit@student.hogent.be};
  \textsuperscript{2} \href{mailto:jan.cannaerts@socmut.be}{jan.cannaerts@socmut.be};
}

%---------- Abstract ----------------------------------------------------------
\Abstract{De toekomst van de mainframe computer of ook 'Big iron' genoemd is iets waar duizenden meningen over bestaan. Ik wil deze toekomst onderzoeken. Deze meningen komen van zowel mensen die mainframe technologie ontwikkelen, als van mensen die al jaar en dag werken op een mainframe computer. Voor mij is dit een onderwerp dat veel gevoel met zich meebrengt. Ik studeer af als mainframe expert en daarom is het ook een onderzoek naar mijn persoonlijke toekomst. Het is een onderwerp dat ook eens moet onderzocht worden om de bestaande meningen hierover, te ondersteunen of misschien zelfs te ontkrachten. Persoonlijk verwacht ik een veelbelovende en stabiele toekomst voor de mainframe. Deze verwachtingen ga ik proberen te bevestigen door onderzoek te doen naar de innovatie binnen de mainframe wereld, de nood binnen bedrijven en de visie van experten. Wat zal quantum computing betekenen voor mainframe? Wat doet cloud computing met mainframe? Wat doet het tekort aan ontwikkelaars met mainframe? Veel vragen dus waar eindelijk een antwoord op moet komen.
}

%---------- Onderzoeksdomein en sleutelwoorden --------------------------------
% TODO: Sleutelwoorden:
%
% Het eerste sleutelwoord beschrijft het onderzoeksdomein. Je kan kiezen uit
% deze lijst:
%
% - Mobiele applicatieontwikkeling
% - Webapplicatieontwikkeling
% - Applicatieontwikkeling (andere)
% - Systeembeheer
% - Netwerkbeheer
% - Mainframe
% - E-business
% - Databanken en big data
% - Machineleertechnieken en kunstmatige intelligentie
% - Andere (specifieer)
%
% De andere sleutelwoorden zijn vrij te kiezen

\Keywords{Onderzoeksdomein. Mainframe --- Quantum computing --- Toekomstvisie} % Keywords
\newcommand{\keywordname}{Sleutelwoorden} % Defines the keywords heading name

%---------- Titel, inhoud -----------------------------------------------------

\begin{document}

\flushbottom % Makes all text pages the same height
\maketitle % Print the title and abstract box
\tableofcontents % Print the contents section
\thispagestyle{empty} % Removes page numbering from the first page

%------------------------------------------------------------------------------
% Hoofdtekst
%------------------------------------------------------------------------------

% De hoofdtekst van het voorstel zit in een apart bestand, zodat het makkelijk
% kan opgenomen worden in de bijlagen van de bachelorproef zelf.
%---------- Inleiding ---------------------------------------------------------

\section{Introductie} % The \section*{} command stops section numbering
\label{sec:introductie}
Mijn onderzoek zal gaan over de toekomst van mainframe. Doordat ik zelf in de wereld van mainframe ben terecht gekomen, word ik vaak geconfronteerd
met verschillende problemen die beginnen op te duiken. Dit gaf mij dan ook het idee om hierover een onderzoek te starten. De volgende problemen zijn op dit moment de voornaamste die zich voordoen:
\begin{itemize}
  \item het tekort aan COBOL/PL1 programmeurs of mensen met de juiste skills
  \item de migratie van mainframe naar cloud computing
  \item de modernisering van z/OS
  \item het updaten van 'legacy' mainframe afhankelijke applicaties
  \item de onderhoudskosten van een mainframe
\end{itemize}
Al deze problemen vormen een bedreiging voor het verder bestaan van de mainframe. In dit onderzoek ga ik proberen elke probleem individueel te verdelen in kleinere deelproblemen en zo dieper te kunnen graven. Mijn doel is om de resultaten van deze onderzoeken te gaan vergelijken met de meningen van zowel experts binnen de ontwikkeling zelf bij IBM, de mensen die dagelijks werken op mainframe en de managers van grote bedrijven zonder IT kennis voor de financiële kant eens te bekijken. De volgende onderzoeksvragen zijn voor deze problemen het meest relevant:
\begin{itemize} 
  \item Is er een tekort aan mensen met de juiste skills?
  \item Welke ontwikkelingen vinden er nu nog plaats in de mainframe wereld?
  \item Hoe zal de mainframe zich verdedigen tegen de opkomst van cloud computing?
  \item Welk effect zal de doorbraak van quantum computing hebben op de mainframe technologie die we vandaag kennen?
  \item Wat is het gemiddelde kostenplaatje van een mainframe binnen een bedrijf?
\end{itemize}
Als het onderzoek een duidelijk en gestaafd antwoord geeft op elk van deze vragen, dan pas zullen we een klaar en duidelijk antwoord kunnen geven op 
de titel van deze bachelorproef.

%---------- Stand van zaken ---------------------------------------------------

\section{Huidige situatie}
\label{sec:state-of-the-art}

Hier beschrijf je de \emph{state-of-the-art} rondom je gekozen onderzoeksdomein. Dit kan bijvoorbeeld een literatuurstudie zijn. Je mag de titel van deze sectie ook aanpassen (literatuurstudie, stand van zaken, enz.). Zijn er al gelijkaardige onderzoeken gevoerd? Wat concluderen ze? Wat is het verschil met jouw onderzoek? Wat is de relevantie met jouw onderzoek?

Verwijs bij elke introductie van een term of bewering over het domein naar de vakliteratuur, bijvoorbeeld~\autocite{Doll1954}! Denk zeker goed na welke werken je refereert en waarom.

% Voor literatuurverwijzingen zijn er twee belangrijke commando's:
% \autocite{KEY} => (Auteur, jaartal) Gebruik dit als de naam van de auteur
%   geen onderdeel is van de zin.
% \textcite{KEY} => Auteur (jaartal)  Gebruik dit als de auteursnaam wel een
%   functie heeft in de zin (bv. ``Uit onderzoek door Doll & Hill (1954) bleek
%   ...'')

Je mag gerust gebruik maken van subsecties in dit onderdeel.

%---------- Methodologie ------------------------------------------------------
\section{Methodologie}
\label{sec:methodologie}

Hier beschrijf je hoe je van plan bent het onderzoek te voeren. Welke onderzoekstechniek ga je toepassen om elk van je onderzoeksvragen te beantwoorden? Gebruik je hiervoor experimenten, vragenlijsten, simulaties? Je beschrijft ook al welke tools je denkt hiervoor te gebruiken of te ontwikkelen.

%---------- Verwachte resultaten ----------------------------------------------
\section{Verwachte resultaten}
\label{sec:verwachte_resultaten}

Hier beschrijf je welke resultaten je verwacht. Als je metingen en simulaties uitvoert, kan je hier al mock-ups maken van de grafieken samen met de verwachte conclusies. Benoem zeker al je assen en de stukken van de grafiek die je gaat gebruiken. Dit zorgt ervoor dat je concreet weet hoe je je data gaat moeten structureren.

%---------- Verwachte conclusies ----------------------------------------------
\section{Verwachte conclusies}
\label{sec:verwachte_conclusies}

Hier beschrijf je wat je verwacht uit je onderzoek, met de motivatie waarom. Het is \textbf{niet} erg indien uit je onderzoek andere resultaten en conclusies vloeien dan dat je hier beschrijft: het is dan juist interessant om te onderzoeken waarom jouw hypothesen niet overeenkomen met de resultaten.



%------------------------------------------------------------------------------
% Referentielijst
%------------------------------------------------------------------------------
% TODO: de gerefereerde werken moeten in BibTeX-bestand ``voorstel.bib''
% voorkomen. Gebruik JabRef om je bibliografie bij te houden en vergeet niet
% om compatibiliteit met Biber/BibLaTeX aan te zetten (File > Switch to
% BibLaTeX mode)

\phantomsection
\printbibliography[heading=bibintoc]

\end{document}
