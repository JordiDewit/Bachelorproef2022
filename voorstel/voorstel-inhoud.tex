%---------- Inleiding ---------------------------------------------------------

\section{Introductie} % The \section*{} command stops section numbering
\label{sec:introductie}
Mijn onderzoek zal gaan over de toekomst van mainframe. Doordat ik zelf in de wereld van mainframe ben terecht gekomen, word ik vaak geconfronteerd
met verschillende problemen die beginnen op te duiken. Dit gaf mij dan ook het idee om hierover een onderzoek te starten. De volgende problemen zijn op dit moment de voornaamste die zich voordoen:
\begin{itemize}
  \item het tekort aan COBOL/PL1 programmeurs of mensen met de juiste skills
  \item de migratie van mainframe naar cloud computing
  \item de modernisering van z/OS
  \item het updaten van 'legacy' mainframe afhankelijke applicaties
  \item de onderhoudskosten van een mainframe
\end{itemize}
Al deze problemen vormen een bedreiging voor het verder bestaan van de mainframe. In dit onderzoek ga ik proberen elke probleem individueel te verdelen in kleinere deelproblemen en zo dieper te kunnen graven. Mijn doel is om de resultaten van deze onderzoeken te gaan vergelijken met de meningen van zowel experts binnen de ontwikkeling zelf bij IBM, de mensen die dagelijks werken op mainframe en de managers van grote bedrijven zonder IT kennis voor de financiële kant eens te bekijken. De volgende onderzoeksvragen zijn voor deze problemen het meest relevant:
\begin{itemize} 
  \item Is er een tekort aan mensen met de juiste skills?
  \item Welke ontwikkelingen vinden er nu nog plaats in de mainframe wereld?
  \item Hoe zal de mainframe zich verdedigen tegen de opkomst van cloud computing?
  \item Welk effect zal de doorbraak van quantum computing hebben op de mainframe technologie die we vandaag kennen?
  \item Wat is het gemiddelde kostenplaatje van een mainframe binnen een bedrijf?
\end{itemize}
Als het onderzoek een duidelijk en gestaafd antwoord geeft op elk van deze vragen, dan pas zullen we een klaar en duidelijk antwoord kunnen geven op 
de titel van deze bachelorproef.

%---------- Stand van zaken ---------------------------------------------------

\section{Huidige situatie}
\label{sec:state-of-the-art}

Hier beschrijf je de \emph{state-of-the-art} rondom je gekozen onderzoeksdomein. Dit kan bijvoorbeeld een literatuurstudie zijn. Je mag de titel van deze sectie ook aanpassen (literatuurstudie, stand van zaken, enz.). Zijn er al gelijkaardige onderzoeken gevoerd? Wat concluderen ze? Wat is het verschil met jouw onderzoek? Wat is de relevantie met jouw onderzoek?

Verwijs bij elke introductie van een term of bewering over het domein naar de vakliteratuur, bijvoorbeeld~\autocite{Doll1954}! Denk zeker goed na welke werken je refereert en waarom.

% Voor literatuurverwijzingen zijn er twee belangrijke commando's:
% \autocite{KEY} => (Auteur, jaartal) Gebruik dit als de naam van de auteur
%   geen onderdeel is van de zin.
% \textcite{KEY} => Auteur (jaartal)  Gebruik dit als de auteursnaam wel een
%   functie heeft in de zin (bv. ``Uit onderzoek door Doll & Hill (1954) bleek
%   ...'')

Je mag gerust gebruik maken van subsecties in dit onderdeel.

%---------- Methodologie ------------------------------------------------------
\section{Methodologie}
\label{sec:methodologie}

Hier beschrijf je hoe je van plan bent het onderzoek te voeren. Welke onderzoekstechniek ga je toepassen om elk van je onderzoeksvragen te beantwoorden? Gebruik je hiervoor experimenten, vragenlijsten, simulaties? Je beschrijft ook al welke tools je denkt hiervoor te gebruiken of te ontwikkelen.

%---------- Verwachte resultaten ----------------------------------------------
\section{Verwachte resultaten}
\label{sec:verwachte_resultaten}

Hier beschrijf je welke resultaten je verwacht. Als je metingen en simulaties uitvoert, kan je hier al mock-ups maken van de grafieken samen met de verwachte conclusies. Benoem zeker al je assen en de stukken van de grafiek die je gaat gebruiken. Dit zorgt ervoor dat je concreet weet hoe je je data gaat moeten structureren.

%---------- Verwachte conclusies ----------------------------------------------
\section{Verwachte conclusies}
\label{sec:verwachte_conclusies}

Hier beschrijf je wat je verwacht uit je onderzoek, met de motivatie waarom. Het is \textbf{niet} erg indien uit je onderzoek andere resultaten en conclusies vloeien dan dat je hier beschrijft: het is dan juist interessant om te onderzoeken waarom jouw hypothesen niet overeenkomen met de resultaten.

