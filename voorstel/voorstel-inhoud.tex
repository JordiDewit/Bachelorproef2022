%---------- Inleiding ---------------------------------------------------------

\section{Introductie} % The \section*{} command stops section numbering
\label{sec:introductie}
Mijn onderzoek zal gaan over de toekomst van mainframe. Doordat ik zelf in de wereld van mainframe ben terecht gekomen, word ik vaak geconfronteerd
met verschillende problemen die op duiken. Dat gaf mij dan ook het idee om hierover een onderzoek te starten. De volgende problemen zijn op dit moment de voornaamste die zich voordoen:
\begin{itemize}
  \item het tekort aan COBOL en PL/1 programmeurs of personen met de juiste skills
  \item de migratie van mainframe naar cloud computing
  \item de modernisering van z/OS
  \item het updaten van 'legacy' mainframe afhankelijke applicaties
  \item de onderhoudskosten van een mainframe
\end{itemize}
Al deze problemen vormen een bedreiging voor het verder bestaan van de mainframe. In dit onderzoek ga ik proberen elke probleem individueel te verdelen in kleinere deelproblemen om zo dieper te kunnen graven. Mijn doel is om de resultaten van deze onderzoeken te gaan vergelijken met de meningen van zowel experts binnen de ontwikkeling bij IBM, de mensen die dagelijks werken op mainframe en de managers van grote bedrijven zonder IT kennis. Dit is namelijk belangrijk om ook de financiële kant te bekijken. De volgende onderzoeksvragen zijn voor deze problemen het meest relevant:
\begin{itemize} 
  \item Is er een tekort aan mensen met de juiste skills?
  \item Welke ontwikkelingen vinden er nu nog plaats in de mainframe wereld?
  \item Hoe zal de mainframe zich verdedigen tegen de opkomst van cloud computing?
  \item Welk effect zal de doorbraak van quantum computing hebben op de mainframe technologie die we vandaag kennen?
  \item Wat is het gemiddelde kostenplaatje van een mainframe binnen een bedrijf?
\end{itemize}
Als het onderzoek een duidelijk en gestaafd antwoord geeft op elk van deze vragen, dan zullen we een klaar en duidelijk antwoord kunnen geven op 
de titel van deze bachelorproef.

%---------- Stand van zaken ---------------------------------------------------

\section{Sneak peek}
\label{sec:state-of-the-art}

\subsection*{Vergrijzing} 

De huidige situatie rond mensen met de juiste skills om op een mainframe te werken, is meer en meer schaars aan het worden. Er is een hele harde 'war on talent' aan de gang. Dat is te verklaren doordat de personen die al jaar en dag op mainframe werken de pensioenleeftijd benaderen. Echter ook zeker omdat er weinigen afstuderen als COBOL of PL1 programmeur. Dit probleem heeft een verband met het ontstaan van de mainframe. De eerste mainframe, namelijk de S/360 van IBM, werd geïntroduceerd op 7 april 1964 \autocite{Elliot2005}. Het was revolutionair dat er voor het eerst een computer was, die in staat was om allerlei toepassingen te verwerken op een performante manier. Dat is ondertussen al 57 jaar geleden. De ontwikkelaars die opgegroeid zijn met de mainframe, naderen nu in versneld tempo de pensioenleeftijd. Nergens is die probleem zo prominent aanwezig dan in de mainframewereld. In 2002 heeft Meta Group hierover een onderzoek gedaan \autcite{Bakker2006}. Hieruit was gebleken dat meer dan 90 procent van de bevraagde bedrijven, geen beleidsplannen hadden om die vergrijzing op te vangen. Evenees is gebleken dat 50 procent van de mainframespecialisten ouder was dan 50 jaar. Dit onderzoek is ondertussen 19 jaar geleden.

\subsection*{System/360 geschiedenis} 

IBM S/360 is een familie van mainframe systemen. Als dit op 7 april 1964 \autocite{Elliot2005} voor het eerst het daglicht zag, was dit uitzonderlijk revolutionair. Op deze systemen kon een volledige dekking aan toepassingen, van klein tot groot, zowel commercieel als wetenschappelijk op uitgevoerd worden. Binnen de S/360 werden er een hele reeks nieuwe computers ontwikkeld, van groot tot klein, van lage tot hoge prestaties, maar met allemaal dezelfde instructieset. Hierdoor konden gebruikers makkelijk upgraden naar grotere systemen naarmate hun behoefte hiertoe leidde. Hiervoor waren er geen dure en complexe applicatie veranderingen nodig. Compatibiliteit was het sleutelwoord bij de introductie van S/360 systemen. 


%---------- Methodologie ------------------------------------------------------
\section{Methodologie}
\label{sec:methodologie}

Ik ga mijn onderzoek zoveel mogelijk aan de realiteit op het werkveld te koppelen. Dit hoop ik te bereiken door langs te gaan bij de bedrijven in België die werken met mainframe, en tot slot, de plaats waar hij wordt ontwikkeld, IBM in Montpellier. Op deze plaatsen hoop ik de juiste mensen te vinden om zoveel mogelijk informatie te verkrijgen en daaruit de juiste conclusies te kunnen trekken. Met op voorhand opgestelde vragenlijsten zal ik naar deze bedrijven gaan. Bij de mensen van mijn stageplaats zal ik dan werken met Google Forms, die ik zal rondsturen naar het personeel. Ook ga ik mij verdiepen in de geschiedenis van de mainframe door een diepe literatuurstudie te doen en zoveel deel te nemen aan introducties of keynotes van nieuwe technologieën voor de mainframe. Ten slotte zal ik zelf proberen te argumenteren door mijn stageprojecten dat de mainframe zeker nog niet uitstervend is.

Zaken die bevraagd zullen worden:
\begin{itemize} 
\item Gebruiksvriendelijkheid van besturingssysteem
\item Meningen van ervaren programmeurs en systeem administrators
\item Opkomende ontwikkelingen
\item Gevoel over mainframe bij hoger management
\item Kostenberekeningen per jaar voor een mainframe 
\item Beleidsplannen geschoold persoon die pensioenleeftijd bereiken
\item IBM's visie over quantum computing op hun mainframes 
\end{itemize}

Deze zaken kunnen tijdens het onderzoek uitbreiden of aangepast worden. 
%---------- Verwachte resultaten ----------------------------------------------
\section{Verwachte resultaten}
\label{sec:verwachte_resultaten}

Hier beschrijf je welke resultaten je verwacht. Als je metingen en simulaties uitvoert, kan je hier al mock-ups maken van de grafieken samen met de verwachte conclusies. Benoem zeker al je assen en de stukken van de grafiek die je gaat gebruiken. Dit zorgt ervoor dat je concreet weet hoe je je data gaat moeten structureren.

%---------- Verwachte conclusies ----------------------------------------------
\section{Verwachte conclusies}
\label{sec:verwachte_conclusies}

Hier beschrijf je wat je verwacht uit je onderzoek, met de motivatie waarom. Het is \textbf{niet} erg indien uit je onderzoek andere resultaten en conclusies vloeien dan dat je hier beschrijft: het is dan juist interessant om te onderzoeken waarom jouw hypothesen niet overeenkomen met de resultaten.

